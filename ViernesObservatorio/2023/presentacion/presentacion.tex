%\documentclass[mathserif,9pt]{beamer}
\documentclass[mathserif,100pt]{beamer}

%\usepackage{beamerthemesplit}
\usepackage{graphicx}
%\usetheme{JuanLesPins}
%\documentclass{beamer}
\usepackage{multimedia}
%\usepackage{media9}
\usepackage{bm}


\usetheme{lankton-keynote}

\def\be{\begin{equation}}
\def\ee{\end{equation}}
\def\bes{\begin{equation*}}
\def\ees{\end{equation*}}
\def\ba{\begin{eqnarray}}
\def\ea{\end{eqnarray}}
\def\bas{\begin{eqnarray*}}
\def\eas{\end{eqnarray*}}
\def\bi{\begin{itemize}}
\def\ei{\end{itemize}}
\def\bc{\begin{columns}}
\def\ecx{\end{columns}}
\def\bm{\begin{minipage}}
\def\em{\end{minipage}}

% For arrows:
\usepackage{tikz}
\usetikzlibrary{arrows,shapes}
\everymath{\displaystyle}

% For transparent box
\usepackage{transparent}

\newcommand\blfootnote[1]{%
  \begingroup
  \renewcommand\thefootnote{}\footnote{#1}%
  \addtocounter{footnote}{-1}%
  \endgroup
}

%\usepackage[left=1cm, right=1cm, bottom=1cm, top=1cm]{geometry}
%\addtolength{\topmargin}{1pt}
%\addtolength{\headsep}{5pt}
%\addtolength{\oddsidemargin}{-4pt}
%\addtolength{\marginparwidth}{20pt}
%\addtolength{\marginparsep}{-3pt}
%\addtolength{\textwidth}{-5pt}

%\newcommand{\overgang}{\transdissolve}
\newcommand{\overgang}{}

\newcommand*{\thead}[1]{\multicolumn{1}{c}{#1}}

\usepackage{amsmath}
% textpos stuff
\usepackage[absolute]{textpos}
\setlength{\TPHorizModule}{30mm}
\setlength{\TPVertModule}{\TPHorizModule}
\textblockorigin{10mm}{10mm} % start everything near the top-left corner
\setlength{\parindent}{0pt}

% German stuff
%\usepackage[german]{babel}
\usepackage[T1]{fontenc}
%\usepackage[latin1]{inputenc}

%\usepackage[utf8]{inputenc}
%\usepackage[T1]{fontenc}
%\usepackage{ngerman} % which should have the same result as \usepackage[ngerman]{babel}

%\useinnertheme{default}
%\useoutertheme{smoothbars}
% Puts think up and down.
%\usetheme{shadow}
%\useoutertheme{default}

%\usecolortheme[rgb={1.,0.,0.}]{structure} % red
%\usecolortheme[rgb={0.,0.,.54}]{structure} % navy blue
%\usecolortheme[rgb={0.25,0.41,.88}]{structure} % royal blue
%\usecolortheme[rgb={0.,0.,.8}]{structure} % deep blue
%\usecolortheme[rgb={0.69,0.13,.13}]{structure} % deep blue
\usecolortheme[rgb={0.,0.8,0.4}]{structure} % deep blue

\newcommand{\chg}[1]{\textcolor{magenta}{#1}}
\newcommand{\fs}[1]{\textcolor{red}{[\textbf{FS:}~#1]}}
\newcommand{\ab}[1]{\textcolor{red}{[\textbf{Alex:}~#1]}}
\newcommand{\wah}[1]{\textcolor{orange}{[\textbf{Wojtek:}~#1]}}

\newcommand{\tcg}{\textcolor{green}}
\newcommand{\tcr}{\textcolor{red}}
\definecolor{mygreen}{rgb}{0, 0.8, 0.4}
\definecolor{mycyan}{rgb}{0.8, 0.3, 0.}
%\setbeamercovered{dynamic}

%\usecolortheme{sidebartab}
%\usecolortheme{crane}
%\usecolortheme{rose}

% For smooth transitions:
%\usepackage{animate}
%\usepackage{tikz}
%\usetikzlibrary{decorations.markings}
%\newcounter{angle}
%\setcounter{angle}{0}


%====================================
% For movie
\usepackage{multimedia}

\setbeamercolor{background canvas}{bg=black}
\setbeamercolor{normal text}{fg=yellow}
\setbeamercolor{frametitle}{fg=mygreen}
%====================================
\setbeamertemplate{navigation symbols}{} %no nav symbols
\setbeamertemplate{background}
{
    \includegraphics[width=\paperwidth,height=\paperheight]{figures/background.jpg}
}
\title[]{{\huge CORRIENTE ESTELARES}\\ {\LARGE Aquellos r\'ios de estrellas en la Galaxia y el Universo local}}
\author[]{{\bf {\scriptsize Mart\'in Federico Mestre}}}

\institute{
  Instituto de Astrof\'isica de La Plata, CONICET-UNLP, Argentina \\
  Facultad de Ciencias Astron\'omicas y Geof\'isicas de La Plata, UNLP, Argentina\\
  \begin{center}
    \includegraphics[width=0.12\textwidth]{logos/logo_ialp_small.jpg}
    \hspace{1mm} \includegraphics[width=0.19\textwidth]{logos/fcaglp.jpg}
    \hspace{1mm} \includegraphics[width=0.18\textwidth]{logos/dots.jpg}
  \end{center}
}
\date[]{La Charla de los Viernes en el Planetario UNLP}




\begin{document}
\frame{\titlepage}

\frame {\frametitle{DOTS members}\overgang
\begin{columns}
  \begin{column}{0.3\textwidth}
    \vspace{5mm}\\
      \includegraphics[width=0.7\textwidth,angle=0]{fotos/Charly.jpg}\\
      {\scriptsize Carlos Argüelles}\\
      \vspace{5mm}
      \includegraphics[width=0.8\textwidth,angle=0]{fotos/Mariano.jpg}\\
      {\scriptsize Mariano Dom\'inguez}\\
  \end{column}
  \hfill
  \begin{column}{0.3\textwidth}
      \vspace{2mm}\\
      \includegraphics[width=0.9\textwidth,angle=180]{fotos/Daniel.jpg}\\
      {\scriptsize Daniel Carpintero}\\
      \vspace{1cm}
      \includegraphics[width=0.7\textwidth,angle=0]{fotos/Nico.jpg}\\
      {\scriptsize Nicol\'as Maffione}\\
  \end{column}
  \hfill
  \begin{column}{0.3\textwidth}
    \includegraphics[width=0.8\textwidth,angle=0]{fotos/Santi.jpg}\\
    {\scriptsize Santiago Collazo}\\
    \vspace{1cm}
    \includegraphics[width=0.8\textwidth,angle=0]{fotos/Martin.jpg}\\
    {\scriptsize Mart\'in Mestre}\\
  \end{column}
\end{columns}
}

\frame {\frametitle{Modelo simplificado de la Galaxia}\overgang

    \begin{center}
      \includegraphics[width=0.9\textwidth]{figures_viernes/simpleGalaxy.jpg}\\
    \end{center}
}
\frame {\frametitle{La Galaxia posee además otras subestructuras}\overgang

    \begin{center}
      \includegraphics[width=0.9\textwidth]{figures_viernes/HaloSubstructureMW.pdf}\\
    \end{center}
}
\frame {\frametitle{La Galaxia posee además otras subestructuras}\overgang

\begin{center}
  \includegraphics[width=0.6\textwidth]{figures_viernes/SnapshotSimulationStreams.jpg}\\
  {\small Consistente con un Universo jerárquico formado por sucesivas fusiones de galaxias}
\end{center}
}

\frame {\frametitle{Simulación de corriente estelar}\overgang

\begin{center}
  Animación: \textcolor{orange}{GlobularClusterStreamInMWwithSubHalos.mp4} \\
\end{center}
}

\frame {\frametitle{Qué es una corriente estelar?}\overgang
  \begin{minipage}{1.0\textwidth}
  {\small Es el sistema que se forma cuando un sistema auto-gravitante de estrellas (cúmulo
  globular o galaxia enana) es desarmada por las fuerzas gravitatorias de marea  producidas por la galaxia
  anfitriona.\\
  Generalmente un par de brazos de marea son formados, uno que va hacia adelante y otro hacia atras del progenitor.
 }
  \end{minipage}
  \vfill
  \begin{columns}
    \begin{column}{0.6\textwidth}
      \begin{center}
        \includegraphics[width=0.9\textwidth]{figures/stream_drawing.pdf}
      \end{center}
    \end{column}
  \end{columns}
}

\frame {\frametitle{El radio de marea}\overgang

  \begin{columns}
    \begin{column}{0.45\textwidth}
      \begin{center}
        \includegraphics[width=0.9\textwidth]{figures/stream_drawing.pdf}\\
        \vspace{5mm}
        Para órbitas circulares en un potencial con simetría esférica:
        \begin{equation}
         r_t \approx \left(\frac{Gm}{\Omega^2-\frac{\partial^2\Phi}{\partial R^2}}\right)^{1/3} \nonumber
        \end{equation}
      \end{center}
    \end{column}
  \hfill
    \begin{column}{0.45\textwidth}
      \begin{center}
        \includegraphics[width=0.9\textwidth]{figures/fig_pe.pdf}\\
      \end{center}
    \end{column}
  \end{columns}
}

\frame {\frametitle{Arcturus stream}\overgang

    \begin{center}
      \includegraphics[width=0.9\textwidth]{figures_viernes/ArcturusGroup.pdf}\\
    \end{center}
}
\frame {\frametitle{Arcturus stream}\overgang

    \begin{center}
      \includegraphics[width=0.4\textwidth]{figures_viernes/ArcturusGroupPlot.eps}\\
    \end{center}
}
\frame {\frametitle{Stellar streams in the Local Universe}\overgang
%\hspace{-1cm}
  \begin{minipage}[t]{\textwidth}
    \begin{center}
      \includegraphics[width=1.0\textwidth]{figures_gd1/martinez-delgado_fig2.jpg}    \end{center}
  \end{minipage}
    {\tiny Mart{\'i}nez-Delgado D. et al. 2018}

}

\frame {\frametitle{Stellar streams in the Milky Way (Origins)}\overgang
  \begin{minipage}[t]{1.0\textwidth}
    \begin{center}
      \includegraphics[width=1\textwidth]{figures_gd1/fos_dr6_marked.jpg}
    \end{center}
  \end{minipage}

{\tiny Belokurov V. et al. 2006}
}

\frame {\frametitle{Stellar streams in the Galaxy\\
  {\normalsize Atlas of the Milky Way Mergers (Malhan et al. 2022)}}\overgang
  \begin{minipage}[t]{1.0\textwidth}
    \begin{center}
      41 stellar streams comprising 9192 Gaia EDR3 stars\\
      \vspace{0.5cm}
      \includegraphics[width=0.8\textwidth]{figures_gd1/atlas.png}
    \end{center}
  \end{minipage}
}

\frame {\frametitle{Stellar streams in the Galaxy\\
  {\normalsize Galstreams Python Package (Cecilia Mateu 2022) }}\overgang
  \begin{minipage}[t]{1.0\textwidth}
    \begin{center}
      95 stellar stream 5D/6D tracks available in Galstreams\\
      \vspace{0.5cm}
      \includegraphics[width=0.8\textwidth]{figures/fig_all_streams_lib.png}
    \end{center}
  \end{minipage}
}

\frame { \frametitle{Stellar streams in the Computer}\overgang
  \begin{minipage}{0.45\textwidth}
    {\scriptsize
      Tidal features:
      \begin{itemize}
      \item Great circles
      \item Plumes
      \item Shells
      \item Umbrellas
      \item Spikes
      \item Giant clouds
      \end{itemize}
    }
  \end{minipage}
  \hfill
  \begin{minipage}{0.49\textwidth}
    \includegraphics[width=1\textwidth]{figures_gd1/Fig2_martinez.pdf}

    {\tiny Mart{\'i}nez-Delgado D. et al. 2010, Johnston et al. 2008}
  \end{minipage}
  \vfill
  \begin{minipage}{0.45\textwidth}
    \includegraphics[width=1\textwidth]{figures_gd1/martinez-delgado_fig1.jpg}

    {\tiny Martinez-Delgado D. 2018}
  \end{minipage}
  \hfill
  \begin{minipage}{0.45\textwidth}
    {\scriptsize  Expected streams around a Milky Way--like galaxy
      for different surface brightness detection
      limits (mag/arcsec$^2$):
      \begin{itemize}
      \item $\mu_{\rm{A}}=31$
      \item $\mu_{\rm{B}}=25$
      \item $\mu_{\rm{C}}=28$
      \end{itemize}
    }
  \end{minipage}
}

\frame {\frametitle{Palomar 5 stream\\
\vspace{-0.3cm}
  {\tiny (Odenkirchen 2000+)}}\overgang
  \begin{minipage}[t]{1.0\textwidth}
    \begin{center}
      \textcolor{cyan}{A stellar stream that probes the halo and the bar}\\
      \vspace{0.2cm}
      \includegraphics[width=0.45\textwidth]{figures/odenkirchen_fig1.pdf}\hfill
      \hspace{1mm}
      \includegraphics[width=0.45\textwidth]{figures/odenkirchen_fig2.pdf}\\
      \vspace{2mm}
      $d_{CG} \approx 16$ kpc\\
      $d_\odot \approx 20$ kpc\\
      length $\approx 20^\circ$ / 7 kpc\\
      width $\approx 120$ pc\\

    \end{center}
  \end{minipage}
}

\frame {\frametitle{Group projects}\overgang
\begin{itemize}[<+->]

\item Influence of chaos on stream evolution (published)
\item GD-1 stream on a MW with fermionic DM halo (to be submitted)
\item Sagittarius stream on a MW with fermionic DM halo (to be submitted)
\item Applications of fermionic DM code to other astrophysical phenomenae (to be submitted)
\end{itemize}
}

\frame {\frametitle{The GD-1 stellar stream\\
\vspace{-0.3cm}
  {\tiny (Grillmair \& Dionatos 2006)}}\overgang
  \begin{minipage}[t]{1.0\textwidth}
    \begin{center}
      \textcolor{cyan}{A cold stream traveling through the halo, \\
      shown in self coordinates $\phi_1, \phi_2$}\\
      {\tiny (Price-Whelan \& Bonaca 2018)}\\
      \vspace{0.2cm}
      \includegraphics[width=0.8\textwidth]{figures_gd1/GD-1_PriceW-Bonaca2018.jpg}\\
      \vspace{2mm}
      {\scriptsize $d_\odot\approx$ 10 kpc \\
      length $\approx 100^\circ$ / 10 kpc\\
       width $\approx 12^\circ$ / 30 pc}
    \end{center}
  \end{minipage}
}

\frame {\frametitle{The GD-1 stellar stream\\
\vspace{-0.3cm}
  {\tiny (Grillmair \& Dionatos 2006)}}\overgang
  \begin{minipage}[t]{1.0\textwidth}
    \begin{center}
      \textcolor{cyan}{A cold stream traveling through the halo, \\
      shown in self coordinates $\phi_1, \phi_2$}\\
      {\tiny (Price-Whelan \& Bonaca 2018)}\\
      \vspace{0.2cm}
      \includegraphics[width=0.8\textwidth]{figures_gd1/GD-1_PriceW-Bonaca2018.jpg}\\
      to study a \\
      \textcolor{brown}{System of Self-Gravitating Fermions}\\
      that satisfy a\\
      \textcolor{green}{Maximum Entropy Production Principle} (the RFK/RAR model)
    \end{center}
  \end{minipage}
}
\frame { \frametitle{Previus motivations}\overgang
  \begin{minipage}{0.9\textwidth}
    \begin{center}
      {\scriptsize GD-1 fit to MWPotential2014 with \textcolor{yellow}{axisymmetric} NFW (Malhan et al. 2019).}
      \includegraphics[width=1.0\textwidth]{figures_gd1/MalhanFitGD-1.png}
    \end{center}
  \end{minipage}
  \vfill
  \pause
  \begin{minipage}{0.49\textwidth}
    \begin{center}
      \includegraphics[width=1\textwidth]{figures_gd1/rotation_curves_data.png}
    \end{center}
  \end{minipage}
  \begin{minipage}{0.49\textwidth}
    {\scriptsize RFK model for MW relies on rotation curves.
    Rotation curves depend on the assumption on \textcolor{yellow}{$V_{\mathrm{LSR}}$}.\\
    \textcolor{orange}{Streams probe the acceleration field.}}
  \end{minipage}
}

\frame { \frametitle{The relativistic fermionic King model}\overgang
  \begin{minipage}{0.9\textwidth}
    \begin{center}
     {\scriptsize We solve Einstein equations for a gas of fermions at finite T in hydrostatic equilibrium
     (i.e. T.O.V), in spherical symmetry Argüelles, Krut, Rueda, Ruffini (2018).}\\

    {\scriptsize
    \begin{equation}
      \frac{d\nu}{d\zeta}=\frac{1}{2}\left[e^{z}+e^{2\zeta}\frac{P(r)}{\rho_{rel}c^2}\right][1-e^{z}]^{-1}
    \end{equation}
    }
    {\scriptsize
    \begin{equation}
      \frac{dz}{d\zeta}=-1+e^{(2\zeta-z)}\frac{\rho(r)}{\rho_{rel}}
    \end{equation}
    }
    {\scriptsize $\zeta=\ln(r/R)$,  $z=\ln\left(\frac{M(r)}{M}\frac{R}{r}\right)$}\\
    {\scriptsize
    \begin{equation}
      \rho(r)=\frac{4\rho_{\rm rel}}{\sqrt{\pi}}\int^\infty_1\epsilon^2[\epsilon^2-1]^{1/2}f(r,\epsilon)d\epsilon
    \end{equation}
    \begin{equation}
      P(r)=\frac{4\rho_{\rm rel}c^2}{3\sqrt{\pi}}\int^\infty_1[\epsilon^2-1]^{3/2}f(r,\epsilon)d\epsilon
    \end{equation}
    }


  \end{center}
\end{minipage}

}
\frame { \frametitle{The RFK model}\overgang
  \begin{minipage}{0.9\textwidth}
    \begin{center}
  Maximum Entropy production Priniciple: Chavanis, MNRAS (1998)
      \begin{equation}
        f(r,\epsilon)=
        \begin{cases}
          \frac{1-e^{[\epsilon-\epsilon_c(r)]/\beta(r)}}
          {1+e^{[\epsilon-\alpha(r)]/\beta(r)}}\quad \epsilon \leq \epsilon_c(r)\\
          0\quad \epsilon > \epsilon_c(r)
        \end{cases}
      \end{equation}

      Klein, Tolman and energy conservation equations imply:
  \begin{equation}
    \alpha(r)=\alpha_0 e^{-(\nu-\nu_0)},\qquad  \beta(r)=\beta_0 e^{-(\nu-\nu_0)},\qquad  \epsilon_c(r)=\epsilon_0 e^{-(\nu-\nu_0)}.
  \end{equation}

    4 RFK parameters:\\
    $m$,\quad $\beta_0=kT_0/mc^2$,\\
    $\theta_0=(\alpha_0-1)/\beta_0$,\quad $W_0=(\epsilon_{c0}-1)/\beta_0$
    \end{center}
  \end{minipage}

}

\frame {\frametitle{A single objective}\overgang

\textcolor{green}{Fit the fermionic model according to a few observables (next slide)}.\\
\vspace{0.5cm}


To begin with we have the following full stream model:
{\scriptsize
  \begin{itemize}[<+->]
    \item Stream(17) = Potential(12) + Orbit\_IC(5).
    \item Potential(12) = Baryons(8) + DM(4).
    \item Baryons(8) = Plummer\_Bulge(2) + 2$\times$ MiyamotoNagai\_Disk(3).
  \end{itemize}
}
\pause
But there is some \textcolor{cyan}{a priori knowledge}:
{\scriptsize
      \begin{itemize}[<+->]
        \item Fits to the orbits of Sagittarius A$^*$-stars imply\\
          \boldsymbol{\textcolor{cyan}{$m_f\geq 56$ keV $\&$ Core Mass $\approx 3.5\times10^6 M_\odot$}}
          (Becerra-Vergara+21). \\
          \boldsymbol{\textcolor{yellow}{We fixed $m_f=56$ keV}}
        \item Fits to the \textcolor{cyan}{radial surface denstiy profiles} and
          \textcolor{cyan}{vertical density profile at solar radius} determine
          the \textcolor{yellow}{Baryonic parameters} (Pouliasis+17).
      \end{itemize}
}
\pause
Then we are left with:
{\scriptsize
  \begin{itemize}[<+->]
    \item Stream(8) = RFK($\theta_0$, $W_0$, $\beta_0$) + Orbit\_IC(5)
  \end{itemize}
}
}

\frame { \frametitle{GD-1 Observables\\
\vspace{-0.6cm}
  {\tiny (Ibata+20)}}\overgang
  \begin{minipage}{0.45\textwidth}
    \includegraphics[width=1.0\textwidth]{figures_gd1/observables1_Ibata2020.jpg}\\
  \end{minipage}
  \hfill
  \begin{minipage}{0.45\textwidth}
    \includegraphics[width=1\textwidth]{figures_gd1/observables2_Ibata2020.jpg}
  \end{minipage}
  \vfill
  \begin{minipage}{0.3\textwidth}
    \includegraphics[width=1\textwidth]{figures_gd1/observables3_Ibata2020.jpg}
  \end{minipage}
  \hfill
  \begin{minipage}{0.65\textwidth}
    {\tiny
      \begin{itemize}[<+->]
        \item STREAMFINDER (Gaia DR2): \textcolor{cyan}{811 candidate members}.
        \item Cross correlation with spectroscopy: \textcolor{pink}{156 stars in RV sample}.
        \item Velocity fit: \textcolor{pink}{117 RV member stars}.
        \item Sky coords fit (\textcolor{pink}{with 117 RV members}): $\phi_2 = S(\phi_1)$.
        \item Final selection ($|\phi_2-S(\phi_1)|<0.6^\circ$ \& not RV outlier): \textcolor{cyan}{603 Gaia stars}.
        \item STREAMFINDER $+$ Pan-STARRS $+$ stellar population models:
          \textcolor{yellow}{Photometric Heliocentric Distance ($D_{\rm{hel}}$)}.
        \item Distance fit (\textcolor{pink}{with 117 RV members}):\textcolor{yellow}{$D_{\rm{hel}}=D(\phi_1)$}.
      \end{itemize}
    }
  \end{minipage}
}

\frame { \frametitle{Fitting procedure}\overgang
  \begin{minipage}{1.0\textwidth}
      \begin{itemize}[<+->]
        \item Define a grid in the domain of Ibata Polynomials $\phi_1$.
        \item $\chi^2_{\eta} = \frac{1}{\sigma_\eta^2}\sum_{i=1}^{100} \Big(\eta^{(i)}-\eta(\phi_1^{(i)})\Big)^2$
        \item $\chi^2_{\rm{stream}} = \chi^2_{\phi_2} + \chi^2_{D}+ \chi^2_{\tilde{\mu}_\alpha}+ \chi^2_{\mu_\delta}+\chi^2_{v_h}$
        \item Use the potential fitted by Malhan+19 to find the
        best fit orbit and use its initial condition as an approximation to our problem.
        \item Now we have a full model stream(3) = RFK($\theta_0$, $W_0$, $\beta_0$) \textcolor{cyan}{effectively}.
        \item Pulish the orbit's initial condition for the RFK model.

      \end{itemize}
  \end{minipage}
}

\frame { \frametitle{Result of the fitting: stream features}\overgang
  {\scriptsize Best fit parameters for 56 keV: \textcolor{green}{$\theta_0 = 36.07$}, \textcolor{green}{$W_0 = 63.41$}, \textcolor{green}{$\beta_0 = 1.26\times10^{-5}$}}
  \begin{columns}
    \begin{column}{0.49\textwidth}
    \vspace{5mm}\\
      \includegraphics[width=1.0\textwidth,angle=0]{figures/observables_position.pdf}\\
      \vspace{5mm}
      \includegraphics[width=1.0\textwidth,angle=0]{figures/observables_pmra.pdf}\\
  \end{column}
  \hfill
    \begin{column}{0.49\textwidth}
      \vspace{5mm}\\
        \includegraphics[width=1.0\textwidth,angle=0]{figures/observables_pmdec.pdf}\\
        \vspace{5mm}
        \includegraphics[width=1.0\textwidth,angle=0]{figures/observables_helvel.pdf}\\
    \end{column}
 \end{columns}
}

\frame { \frametitle{Result of the fitting: stream features}\overgang
  {\scriptsize Best fit parameters for 56 keV: \textcolor{green}{$\theta_0 = 36.07$}, \textcolor{green}{$W_0 = 63.41$}, \textcolor{green}{$\beta_0 = 1.26\times10^{-5}$}}
  \begin{columns}
    \begin{column}{0.49\textwidth}
    \vspace{5mm}\\
      \includegraphics[width=1.0\textwidth,angle=0]{figures/observables_position.pdf}\\
      \vspace{5mm}
      \includegraphics[width=1.0\textwidth,angle=0]{figures/observables_pmra.pdf}\\
  \end{column}
  \hfill
    \begin{column}{0.49\textwidth}
      \vspace{5mm}\\
        \includegraphics[width=1.0\textwidth,angle=0]{figures/observables_pmdec.pdf}\\
        \vspace{5mm}
        \includegraphics[width=1.0\textwidth,angle=0]{figures/observables_heldist.pdf}\\
    \end{column}
 \end{columns}
}

\frame { \frametitle{Result of the fitting: rotation curve}\overgang
 {\scriptsize Best fit parameters for 56 keV: \textcolor{green}{$\theta_0 = 36.07$}, \textcolor{green}{$W_0 = 63.41$}, \textcolor{green}{$\beta_0 = 1.26\times10^{-5}$}}
  \begin{minipage}{0.9\textwidth}
    \includegraphics[width=1\textwidth]{figures/rotation_curves.pdf}
  \end{minipage}
  \pause
  \vfill
  \begin{minipage}{0.8\textwidth}
    Important halo features:\\
    \textcolor{yellow}{
    Radius = 28 kpc \hspace{2cm}
    Mass = 1.4$\times$ 10$^{11}$ $M_\odot$}
  \end{minipage}
}

\frame { \frametitle{Going to higher fermion masses to allow more compact cores}\overgang
  \begin{minipage}{0.9\textwidth}
    \includegraphics[width=1\textwidth]{figures/density_profiles.pdf}
  \end{minipage}
}


\frame { \frametitle{Conclusion of the GD-1 project}\overgang
\begin{minipage}{1.0\textwidth}
  \begin{center}
    \begin{itemize}[<+->]
    \item It is possible to model de GD-1 stream using a
    dark matter core-halo \textcolor{cyan}{spherical} distribution constituted by
    self-gravitating fermions.
    \item Stellar streams are good tracers of the \textcolor{cyan}{acceleration} field.
    \end{itemize}
  \end{center}
\end{minipage}
}

\frame {\frametitle{Present and future projects}\overgang
\begin{itemize}[<+->]
\item GalacticDynamics.jl
\item Multiple streams embedded in a MW with fermionic DM halo
\item Acceleration field measurement independently of the MW DM model
\item Community Atlas of Tidal Streams (\href{https://stellarstreams.org/streams22/}{\beamergotobutton{CATS}})
\item Dark Energy Science Collaboration (DESC-LSST)
\end{itemize}
}

\frame { \frametitle{¡Muchas gracias!}\overgang
\begin{center}
  \includegraphics[width=0.5\textwidth,angle=90]{logos/dots_Luisi.jpg}
\end{center}
\vfill
\begin{minipage}{
\textwidth}
  \begin{center}
    \includegraphics[width=0.25\textwidth]{logos/fcaglp}
     \includegraphics[width=0.15\textwidth]{logos/logo_ialp_small.jpg}
     \includegraphics[width=0.35\textwidth]{figures_gd1/ccad.png}
     \includegraphics[width=0.2\textwidth]{logos/dirac.png}
  \end{center}
\end{minipage}
}


\end{document}
